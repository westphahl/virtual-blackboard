\section{Server}

\subsection{Programmabläufe}

\subsubsection{Hauptprogramm}
\begin{lstlisting}
# Sicherstellen, dass noch kein Server läuft
# Mutex für Tafelzugriff initialisieren
# Mutex für Zugriff auf Client-Liste initialisieren
# Initialisierung der Tafel (Shared Memory)
# Initialisierung der Client-Liste (doppelt verkettete Liste)
# Initialisierung Semaphore (Zähler) für aktive Clients
# Message Queue für Logging initialisieren
# Initialisiere Trigger für Broadcasting-Agent (Condition Variable > pthread)
# Initialisiere Trigger "Tafel archiviert" (Semaphore)
# Signal registrieren für "System beenden"
# Socket für Netzwerkkommunikation öffnen

# Fork: Logger (externes Programm)
# Fork: Archivierer (externes Programm), wenn Debugmodus mit Archivierungsintervall
# Starte Broadcasting-Agent als Thread
# Starte Login-Thread
\end{lstlisting}

\subsubsection{Signal System beenden}
\begin{lstlisting}
# Mutex-Down: Clientliste
    # Kill: Login-Thread
# Mutex-Up: Clientliste
# Trigger Broadcasting Agent: Clients beenden (quit)
# Warte auf Sempahore (Clientzähler) == 0
    # Kill: Broadcasting Agent
# Netzwerksocket schließen
# Kill: Archivierer
# Kill: Logger
# Message Queue (Logger) löschen
# Freigabe: Shared Memory (Tafel)
\end{lstlisting}

\subsubsection{Login-Thread}
\begin{lstlisting}
# Warte auf Login von Client
# Starte für jeden Login Client-Thread
\end{lstlisting}

\subsubsection{Client-Thread}
\begin{lstlisting}
# Mutex-Down für Zugriff auf Client-Liste 
    # Wenn Rolle Dozent: Prüfung, bereits ein Dozent angemeldet?
        # wenn Ja: ändere Rolle Dozent -> Student
        # wenn Nein: Schreibrecht zuweisen
    # Wenn Rolle egal: ändere Rolle zu Student oder Dozent falls noch nicht vorhanden
    # Client-ID, Client-Name, Rolle + Zugriffsrecht in Liste eintragen
    # Semaphore (Clientzähler) Up
# Mutex-Up für Zugriff auf Client-Liste
# Bei Fehler: Fehlernachticht an Client und Client-Thread beenden
# Trigger Broadcasting-Agent: Statusnachricht an Clients
# Warte auf Befehle von verbundenem Client
\end{lstlisting}
\begin{lstlisting}
> quit (Client beenden) bzw. Client schließt Verbindung
    # Mutex-Down für Zugriff auf Client-Liste
        # Client aus Client-Liste austragen
        # Semaphore (Client-Zähler) Down
    # Mutex-Up
    # Trigger Broadcasting-Agent: Sende neue Clientanzahl
    # Tread beenden
\end{lstlisting}
\begin{lstlisting}
> write_request (Schreibrecht-Anfrage (Client -> Server))
    # Mutex-Down für Zugriff auf Client-Liste
        # hat Client bereits schreibrecht?
            # wenn Ja: Fehlermeldung
		# ist Client Dozent?
			# wenn Ja: 
				  # Schreibrecht geht automatisch zum Dozent
				  # Statusnachricht an Clients
        # suche Dozent in Clientliste
            # kein Dozent: Fehlermeldung
    # Mutex-Up
    # Anfrage für Schreibrecht an Dozent
\end{lstlisting}
\begin{lstlisting}
> write_reply (Schreibrecht-Weiterleitung (Server -> Dozenten))
    # Antwort Nein: Fehlermeldung an anfragenden Benutzer
	# Antwort Ja:
		# Mutex-Down für Zugriff auf Client-Liste
			# setze alter Benutzer mit Schreibrechten: keine Schreibrechte
			# aktueller Benutzer: Schreibrecht
		# Mutex-Up
    	# Trigger Broadcasting-Agent: Statusnachricht an alle Clients
\end{lstlisting}
\begin{lstlisting}
> shutdown
    # Mutex-Down für Zugriff auf Client-Liste
        # Benuter ist Dozent?
            # wenn Nein: Fehlermeldung
    # Mutex-Up
    # Signal senden: System beenden
\end{lstlisting}
\begin{lstlisting}
> release
    # Mutex-Down für Zugriff auf Client-Liste
        # Hat Client schreibrecht & nicht Dozent?
            # wenn Nein: Fehlermeldung
        # aktueller Benutzer: Schreibrecht > Nein
        # Dozent: Schreibrecht Ja
    # Mutex-Up
    # Trigger Broadcasting-Agent: Statusnachricht an alle Clients
\end{lstlisting}
\begin{lstlisting}
> acquire
    # Mutex-Down für Zugriff auf Client-Liste
        # aktueller Benutzer ist Dozent & kein Schreibrecht?
            # wenn Nein: Fehlermeldung
        # entziehe Tutor Schreibrecht
        # setze Dozent Schreibrecht
    # Mutex-Up
    # Trigger Broadcasting-Agent: Statusnachricht an alle Clients
\end{lstlisting}
\begin{lstlisting}
> modify
    # Mutex-Down für Zugriff auf Client-Liste
        # Benutzer hat Schreibrechte?
            # wenn Nein: Fehlermeldung
    # Mutex-Up
    # Mutex-Down für Zugriff auf Tafel
        # Änderungen in Shared Memory schreiben
    # Mutex-Up
    # Trigger Broadcasting-Agent: Tafeländerung
\end{lstlisting}
\begin{lstlisting}
> clear
    # Mutex-Down für Zugriff auf Client-Liste
        # Benutzer hat Schreibrechte?
            # wenn Nein: Fehlermeldung
    # Mutex-Up
    # Mutex-Down für Zugriff auf Tafel
    # Trigger Archivierer
        *** Im Archivierer ist kein Mutex-Down notwendig. Dies erfolgt vor
        *** der Triggerung des Archivierers, um sicherzustellen, das in der
        *** Zwischenzeit niemand anders auf die Tafel zugreift.
        *** Der Archivierer macht nach dem Sichern der Tafel einen Mutex-Up
    # Mutex-Down für Zugriff auf Tafel
        # Lösche Tafel  
    # Mutex Up
    # Trigger Broadcasting-Agent: Tafel löschen
\end{lstlisting}

\subsubsection{Broadcasting-Agent}
\begin{lstlisting}
# Warte auf Trigger
# Wenn Tafeländerung
    # Mutex-Down für Zugriff auf Tafel
        # Lese Tafelinhalt
    # Mutex-Up
	# Sendet Tafelinhalt
# Wenn Tafel löschen
	# Sende clear an alle Clients
# Wenn Statusänderung
	# Sende Statusnachricht an alle Clients
\end{lstlisting}

\subsection{Funktionshierarchie}

\subsection{Modulhierarchie}

\subsection{Quellcode}
Der Quellcode ist auf der CD zu finden.