\section{Client}
Der Client stellt eine Verbindung zum Server her. Es werden bein starten des Client die Server- und Logindaten angegeben.

\subsection{Module}
\begin{itemize}
	\item Login
	\item Benutzeroberfläche (GUI)
	\item Live-Agent
	\item Listener-Thread
	\item Netzwerkkommunikation
	\begin{itemize}
		\item Struct vor Senden Serialisieren
		\item Nach Empfang wieder umwandeln
	\end{itemize}
\end{itemize}

\subsection{Programmstart}
Das Programm wird aus der Konsole gestartet. Es müssen folgende Parameter angegeben werden:
\begin{itemize}
	\item Servername oder IP (Servername wird in IP umgewandelt)
	\item Port
	\item Benutzername
	\item Rolle
\end{itemize}

Dabei können Benutzername und Rolle frei gewählt werden. Ist der Benutzername schon vergeben, wird ... . Ist die angegebene Rolle schon belegt, wird der Benutzer automatisch als Student eingetragen.

\paragraph{Beispiel:}
\code{> ./client 127.0.0.1 8080 michael student}

\subsection{Strukturen (intern)}
\subsubsection{Benutzerrollen}
\begin{lstlisting}
enum ROLE {
    Dozent = 1
    Tutor = 2
    Student = 3
};
\end{lstlisting}

\subsubsection{Schreib- und Leserecht}
Als Datentyp wird \code{uint8\_t} verwendet.
\begin{itemize}
	\item Modify 0 = nur lesend
	\item Modify 1 = schreibzugriff (exclusiv)
\end{itemize}

\subsection{Abläufe \label{Abläufe}}

\subsubsection{Programmstart}
\begin{lstlisting}
# Start mit Parameter (Server-IP, Port, Username, Rolle)
# Socket für Netzwerkkommunikation öffnen
# Logindaten (Username und gewünschte Rolle) an Server senden
# Login erfolgreich?
    # wenn NEIN: 
        # Fehlermeldung ausgeben
        # Kill: Client
    # wenn JA:
        # Userdaten und -rechte speichern (ID, Name, Rechte)
# Mutex für lokalen Tafelzugriff initialisieren (gesperrt)
# Initialisiere lokale Tafel
# Starte Command-Thread
# Starte Live-Agent
# Starte GUI
# Starte Listener-Thread
# Starte Trigger für Live-Agent
    # Mutex-Down für lokale Tafel
    # Fordert aktuellen Tafelinhalt an
    # Mutex-Up
\end{lstlisting}

\subsubsection{Command-Thread}
\begin{lstlisting}
> quit (Client beenden)
    # Sende Befehl "quit" an den Server
    # Mutex-Down für lokale Tafel
    # Beende Trigger für Live-Agent
    # Kill: Listener-Thread
    # Kill: GUI
    # Kill: Live-Agent
    # Kill: Command-Thread
    # Lösche lokale Tafel
    # Lösche Mutex für Tafelzugriff

> request (Schreibrecht anfordern)
    # Ist Client Student?
        # Wenn JA:
            # Sende Befehl "request" an den Server
    # Ist Client Dozent?
        # Wenn JA:
            # Dialog ob Benutzer schreibrecht bekommen soll
            # Sende Antwort an Server
    # Schreibrecht erteilt?
        # Wenn JA:
            # Deaktiviere Button 'Schreibrecht anfordern'
            # Aktiviere Button 'Schreibrecht abgeben'
            # Schreibrecht auf lokale Tafel gewähren
        # Wenn NEIN:
            # Hinweis das Anfrage abgeleht wurde.

> shutdown (System beenden)
    # Ist Client Dozent?
        # Wenn JA:
            # Sende Befehl "shutdown" an den Server

> release (Schreibrecht abgeben)
    # Ist Client Tutor?
        # Wenn JA:
            # Sende Befehl "release" an den Server

> acquire (Schreibrecht entziehen)
    # Ist Client Dozent?
        # Wenn JA:
            # Sende Befehl "acquire" an den Server

> clear (Tafel löschen)
    # Hat Client Schreibrechte?
        # Wenn JA:
            # Sende Befehl "clear" an den Server
\end{lstlisting}

\subsubsection{Live-Agent}
\begin{lstlisting}
> modify (Tafel ändern)
    # Hat Client schreibrecht?
        # Wenn JA:
            # Mutex-Down für lokale Tafel
            # Ist Tafel voll?
                # Wenn JA:
                    # Fehlermeldung
                # Wenn NEIN:
                    # Schreibe Änderung in lokale Tafel
            # Mutex-Up
            # Trigger für Tafel starten.
    # Trigger für Tafel sendet dann die Daten in bestimmten Intervallen.
    # Trigger abgelaufen?
        # Wenn JA:
            # Mutex-Down für lokale Tafel
            # Sende Tafel an Server
            # Erfolgreiche Sendung?
                # Wenn NEIN:
                    # Tafel nochmals senden
            # Mutex-Up
\end{lstlisting}

\subsubsection{GUI}
\begin{lstlisting}
# // Tafel wird als GtkTextView gespeichert.
# GtkTextBuffer *gtkbuf = gtk_text_view_get_buffer(textview);

# GtkTextIter startIter, endIter;
# char *mybuf;

# gtk_text_buffer_get_start_iter(gtkbuf, &startIter);
# gtk_text_buffer_get_end_iter(gtkbuf, &endIter);

# // Speichern in char*
# mybuf = gtk_text_buffer_get_text(gtkbuf, &startIter, &endIter, FALSE);

# // Tafel leeren
# gtk_text_buffer_set_text(gtkbuf, "", -1);

# // Tafel wieder befüllen
# gtk_text_buffer_set_text(gtkbuf, mybuf, -1);
\end{lstlisting}

\subsubsection{Listener-Thread}
\begin{lstlisting}
# Wartet auf Nachrichten vom Server (Broadcasting-Thread)
# Aktuallisierung der lokalen Tafel und der Statusinformationen.

> board_modified (Tafel-Update)
    # Mutex-Down für lokale Tafel
    # Tafel aktuallisieren
    # Mutex-Up

> states_changed (Statusänderung)
    # GUI-Informationen aktuallisieren

> my_state_changed (eigene Rechte bekommen/entzogen)
    # Schreibrecht erhalten?
        # Wenn JA:
            # Button "Schreibrecht anfordern" deaktivieren
            # Tafel editierbar setzten
    # Schreibrecht abgegeben/entzogen?
        # Wenn JA:
            # Tafel nicht-editierbar setzten
            # Button "Schreibrecht anfordern" aktivieren
\end{lstlisting}

\subsection{Tafel-Trigger}
Wenn auf die Tafel geschrieben wird, dann wird ein Timeout-Signal gestartet. Wenn dieses abgelaufen ist, wird die Tafel an den Server gesendet und somit an alle Clients verteilt.
Bei jeder Änderung wird der Timeout zurückgesetzt. Wenn der Timeout 5x zurückgesetzt wurde, dann wird die Tafel dennoch zum Server gesendet und der Timeout-Counter zurückgesetzt.
\begin{lstlisting}
# Tafel wird geändert
    # Timeout (200ms) wird (neu) gestartet
    # Timeout-Counter +1
    # Timeout abgelaufen oder Timeout-Counter = 3?
        # Wenn JA:
            # Mutex-Down für lokale Tafel
            # Tafel an Server senden
            # Timeout-Counter = 0
            # Mutex-Up
\end{lstlisting}
