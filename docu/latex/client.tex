\section{Client}

\subsection{Programmabläufe}

\subsubsection{Hauptprogramm}
\begin{lstlisting}
# Start mit Parameter (Server-IP, Port, Username, Rolle)
# Socket für Netzwerkkommunikation öffnen
# Mutex für lokalen Tafelzugriff initialisieren (gesperrt)
# Initialisiere lokale Tafel
# Starte Listener-Thread
# Logindaten (Username und gewünschte Rolle) an Server senden
# Login erfolgreich?
    # wenn NEIN: 
        # Fehlermeldung ausgeben
        # Kill: Client
    # wenn JA:
        # Userdaten und -rechte speichern (ID, Name, Rechte)
# Starte Command-Thread
# Starte Live-Agent
# Starte GUI
# Starte Trigger für Live-Agent
    # Mutex-Down für lokale Tafel
    # Fordert aktuellen Tafelinhalt an
    # Mutex-Up
\end{lstlisting}

\subsubsection{Command-Thread}
\begin{lstlisting}
> quit (Client beenden)
    # Sende Befehl "quit" an den Server
    # Mutex-Down für lokale Tafel
    # Beende Trigger für Live-Agent
    # Kill: Listener-Thread
    # Kill: GUI
    # Kill: Live-Agent
    # Kill: Command-Thread
    # Lösche lokale Tafel
    # Lösche Mutex für Tafelzugriff
\end{lstlisting}
\begin{lstlisting}
> request (Schreibrecht anfordern)
    # Ist Client Student?
        # Wenn JA:
            # Sende Befehl "request" an den Server
    # Ist Client Dozent?
        # Wenn JA:
            # Dialog ob Benutzer schreibrecht bekommen soll
            # Sende Antwort an Server
    # Schreibrecht erteilt?
        # Wenn JA:
            # Deaktiviere Button 'Schreibrecht anfordern'
            # Aktiviere Button 'Schreibrecht abgeben'
            # Schreibrecht auf lokale Tafel gewähren
        # Wenn NEIN:
            # Hinweis das Anfrage abgeleht wurde.
\end{lstlisting}
\begin{lstlisting}
> shutdown (System beenden)
    # Ist Client Dozent?
        # Wenn JA:
            # Sende Befehl "shutdown" an den Server
\end{lstlisting}
\begin{lstlisting}
> release (Schreibrecht abgeben)
    # Ist Client Tutor?
        # Wenn JA:
            # Sende Befehl "release" an den Server
\end{lstlisting}
\begin{lstlisting}
> acquire (Schreibrecht entziehen)
    # Ist Client Dozent?
        # Wenn JA:
            # Sende Befehl "acquire" an den Server
\end{lstlisting}
\begin{lstlisting}
> clear (Tafel löschen)
    # Hat Client Schreibrechte?
        # Wenn JA:
            # Sende Befehl "clear" an den Server
\end{lstlisting}

\subsubsection{Live-Agent}
\begin{lstlisting}
> modify (Tafel ändern)
    # Hat Client schreibrecht?
        # Wenn JA:
            # Mutex-Down für lokale Tafel
            # Ist Tafel voll?
                # Wenn JA:
                    # Fehlermeldung
                # Wenn NEIN:
                    # Schreibe Änderung in lokale Tafel
            # Mutex-Up
            # Trigger für Tafel starten.
    # Trigger für Tafel sendet dann die Daten in bestimmten Intervallen.
    # Trigger abgelaufen?
        # Wenn JA:
            # Mutex-Down für lokale Tafel
            # Sende Tafel an Server
            # Erfolgreiche Sendung?
                # Wenn NEIN:
                    # Tafel nochmals senden
            # Mutex-Up
\end{lstlisting}

\subsubsection{Listener-Thread}
\begin{lstlisting}
# Wartet auf Nachrichten vom Server (Broadcasting-Thread)
# Aktuallisierung der lokalen Tafel und der Statusinformationen.
\end{lstlisting}
\begin{lstlisting}
> board_modified (Tafel-Update)
    # Mutex-Down für lokale Tafel
    # Tafel aktuallisieren
    # Mutex-Up
\end{lstlisting}
\begin{lstlisting}
> states_changed (Statusänderung)
    # GUI-Informationen aktuallisieren
\end{lstlisting}
\begin{lstlisting}
> my_state_changed (eigene Rechte bekommen/entzogen)
    # Schreibrecht erhalten?
        # Wenn JA:
            # Button "Schreibrecht anfordern" deaktivieren
            # Tafel editierbar setzten
    # Schreibrecht abgegeben/entzogen?
        # Wenn JA:
            # Tafel nicht-editierbar setzten
            # Button "Schreibrecht anfordern" aktivieren
\end{lstlisting}

\subsubsection{Tafel-Trigger}
\begin{lstlisting}
# Tafel wird geändert
    # Timeout (200ms) wird (neu) gestartet
    # Timeout-Counter +1
    # Timeout abgelaufen oder Timeout-Counter = 3?
        # Wenn JA:
            # Mutex-Down für lokale Tafel
            # Tafel an Server senden
            # Timeout-Counter = 0
            # Mutex-Up
\end{lstlisting}

\subsection{Funktionshierarchie}

\subsection{Modulhierarchie}

\subsection{Quellcode}
Der Quellcode ist auf der CD zu finden.