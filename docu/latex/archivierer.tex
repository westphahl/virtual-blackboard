\section{Archivierer}

\subsection{Programmablauf}
\begin{lstlisting}
# Öffne Logfile (schreibbar)
# Warte auf Trigger bzw. Ablauf von Timer (Debug-Modus)
# Ausgelöst durch Timer?
    # wenn Ja: Mutex
    *** Vor dem Auslesen der Tafel ist nur ein Mutex-Down notwendig, wenn
    *** der Auslöser für die Archivierung durch den Timer erfolgt ist.
    *** Andernfalls ist dies bereits durch den Client-Thread geschehen um
    *** sicherzustellen, dass der Tafelinhalt erst nach dem Archivieren
    *** gelöscht wird.
# Tafel auslesen
# Mutex-Up
# Zeitstempel + Tafelinhalt in Datei schreiben (blockweise)
\end{lstlisting}

\subsection{Funktionshierarchie}

\subsection{Modulhierarchie}

\subsection{Quellcode}
Der Quellcode ist auf der CD zu finden.