\section{Entwurf der Softwarekomponenten}


\subsection{Funktionalität Client}
Das Programm wird aus der Konsole gestartet. Es müssen folgende Parameter angegeben werden:
\begin{itemize}
	\item -s - Servername oder IP (Servername wird in IP umgewandelt)
	\item -p - Port\footnote{optional, wird sonst auf Default-Port (50000) gesetzt}
	\item -b - Benutzername
	\item -r - Rolle
\end{itemize}

Dabei können Benutzername und Rolle frei gewählt werden. Ist die angegebene Rolle schon belegt, wird der Benutzer automatisch als Student eingetragen.

\paragraph*{Beispiele: \\}
\code{> ./client 127.0.0.1 50100 michael student} \\
\code{> ./client 127.0.0.1 michael dozent}

\subsubsection{Hauptprogramm}
\begin{lstlisting}
# Start mit Parameter (Server-IP, Port, Username, Rolle)
# Socket für Netzwerkkommunikation öffnen
# Logindaten (Username und gewünschte Rolle) an Server senden
# Login erfolgreich?
    # wenn NEIN: 
        # Fehlermeldung ausgeben
        # Kill: Client
    # wenn JA:
        # Userdaten und -rechte speichern (ID, Name, Rechte)
# Mutex für lokalen Tafelzugriff initialisieren (gesperrt)
# Initialisiere lokale Tafel
# Starte Command-Thread
# Starte Live-Agent
# Starte GUI
# Starte Listener-Thread
# Starte Trigger für Live-Agent
    # Mutex-Down für lokale Tafel
    # Fordert aktuellen Tafelinhalt an
    # Mutex-Up
\end{lstlisting}

\subsubsection{Command-Thread}
\begin{lstlisting}
> quit (Client beenden)
    # Sende Befehl "quit" an den Server
    # Mutex-Down für lokale Tafel
    # Beende Trigger für Live-Agent
    # Kill: Listener-Thread
    # Kill: GUI
    # Kill: Live-Agent
    # Kill: Command-Thread
    # Lösche lokale Tafel
    # Lösche Mutex für Tafelzugriff
\end{lstlisting}
\begin{lstlisting}
> request (Schreibrecht anfordern)
    # Ist Client Student?
        # Wenn JA:
            # Sende Befehl "request" an den Server
    # Ist Client Dozent?
        # Wenn JA:
            # Dialog ob Benutzer schreibrecht bekommen soll
            # Sende Antwort an Server
    # Schreibrecht erteilt?
        # Wenn JA:
            # Deaktiviere Button 'Schreibrecht anfordern'
            # Aktiviere Button 'Schreibrecht abgeben'
            # Schreibrecht auf lokale Tafel gewähren
        # Wenn NEIN:
            # Hinweis das Anfrage abgeleht wurde.
\end{lstlisting}
\begin{lstlisting}
> shutdown (System beenden)
    # Ist Client Dozent?
        # Wenn JA:
            # Sende Befehl "shutdown" an den Server
\end{lstlisting}
\begin{lstlisting}
> release (Schreibrecht abgeben)
    # Ist Client Tutor?
        # Wenn JA:
            # Sende Befehl "release" an den Server
\end{lstlisting}
\begin{lstlisting}
> acquire (Schreibrecht entziehen)
    # Ist Client Dozent?
        # Wenn JA:
            # Sende Befehl "acquire" an den Server
\end{lstlisting}
\begin{lstlisting}
> clear (Tafel löschen)
    # Hat Client Schreibrechte?
        # Wenn JA:
            # Sende Befehl "clear" an den Server
\end{lstlisting}

\subsubsection{Live-Agent}
\begin{lstlisting}
> modify (Tafel ändern)
    # Hat Client schreibrecht?
        # Wenn JA:
            # Mutex-Down für lokale Tafel
            # Ist Tafel voll?
                # Wenn JA:
                    # Fehlermeldung
                # Wenn NEIN:
                    # Schreibe Änderung in lokale Tafel
            # Mutex-Up
            # Trigger für Tafel starten.
    # Trigger für Tafel sendet dann die Daten in bestimmten Intervallen.
    # Trigger abgelaufen?
        # Wenn JA:
            # Mutex-Down für lokale Tafel
            # Sende Tafel an Server
            # Erfolgreiche Sendung?
                # Wenn NEIN:
                    # Tafel nochmals senden
            # Mutex-Up
\end{lstlisting}

\subsubsection{Listener-Thread}
\begin{lstlisting}
> board_modified (Tafel-Update)
    # Mutex-Down für lokale Tafel
    # Tafel aktuallisieren
    # Mutex-Up
\end{lstlisting}
\begin{lstlisting}
> states_changed (Statusänderung)
    # GUI-Informationen aktuallisieren
\end{lstlisting}
\begin{lstlisting}
> my_state_changed (eigene Rechte bekommen/entzogen)
    # Schreibrecht erhalten?
        # Wenn JA:
            # Button "Schreibrecht anfordern" deaktivieren
            # Tafel editierbar setzten
    # Schreibrecht abgegeben/entzogen?
        # Wenn JA:
            # Tafel nicht-editierbar setzten
            # Button "Schreibrecht anfordern" aktivieren
\end{lstlisting}

\subsubsection{GUI}
\begin{lstlisting}
# // Tafel wird als GtkTextView gespeichert.
# GtkTextBuffer *gtkbuf = gtk_text_view_get_buffer(textview);

# GtkTextIter startIter, endIter;
# char *mybuf;

# gtk_text_buffer_get_start_iter(gtkbuf, &startIter);
# gtk_text_buffer_get_end_iter(gtkbuf, &endIter);

# // Speichern in char*
# mybuf = gtk_text_buffer_get_text(gtkbuf, &startIter, &endIter, FALSE);

# // Tafel leeren
# gtk_text_buffer_set_text(gtkbuf, "", -1);

# // Tafel wieder befüllen
# gtk_text_buffer_set_text(gtkbuf, mybuf, -1);
\end{lstlisting}

\subsubsection{Tafel-Trigger}
Wenn auf die Tafel geschrieben wird, dann wird ein Timeout-Signal gestartet. Wenn dieses abgelaufen ist, wird die Tafel an den Server gesendet und somit an alle Clients verteilt.
Bei jeder Änderung wird der Timeout zurückgesetzt. Wenn der Timeout 5x zurückgesetzt wurde, dann wird die Tafel dennoch zum Server gesendet und der Timeout-Counter zurückgesetzt.
\begin{lstlisting}
# Tafel wird geändert
    # Timeout (200ms) wird (neu) gestartet
    # Timeout-Counter +1
    # Timeout abgelaufen oder Timeout-Counter = 3?
        # Wenn JA:
            # Mutex-Down für lokale Tafel
            # Tafel an Server senden
            # Timeout-Counter = 0
            # Mutex-Up
\end{lstlisting}


\subsection{Funktionalität Server}
Das Programm wird aus der Konsole gestartet. Es können folgende Parameter angegeben werden:
\begin{itemize}
	\item -p - Port\footnote{optional, sonst wird Default-Port (50000) benutzt.}
	\item -d - Debugging-Mode (ohne Argument)
\end{itemize}

\paragraph*{Beispiele: \\}
\code{> ./server -p 8080 -d} \\
\code{> ./server}

\subsubsection{Hauptprogramm}
\begin{lstlisting}
# Sicherstellen, dass noch kein Server läuft
# Mutex für Tafelzugriff initialisieren (gesperrt)
# Mutex für Zugriff auf Client-Liste initialisieren (gesperrt)
# Initialisierung der Tafel (Shared Memory)
# Initialisierung der Client-Liste (doppelt verkettete Liste)
# Initialisierung Semaphore (Zähler) für aktive Clients (*** GEEIGNET??? ***)
# Message Queue für Logging initialisieren
# Initialisiere Trigger für Broadcasting-Agent (*** IMPLEMENTIERUNG??? ***)
# Initialisiere Trigger "Tafel archiviert" (Condition Variable > pthread)
# Signal registrieren für "System beenden"
# Socket für Netzwerkkommunikation öffnen

# Fork: Logger (externes Programm)
# Fork: Archivierer (externes Programm), wenn Debugmodus mit Archivierungsintervall
# Starte Broadcasting-Agent als Thread
# Starte Login-Thread
# Mutex für Tafelzugriff freigeben
# Mutex für Clientliste freigeben
\end{lstlisting}

\subsubsection{Signal System beenden}
\begin{lstlisting}
# Mutex-Down: Clientliste
    # Kill: Login-Thread
# Mutex-Up: Clientliste
# Trigger Broadcasting Agent: Clients beenden (quit)
# Warte auf Sempahore (Clientzähler) == 0 (*** GEEIGNET??? ***)
    # Kill: Broadcasting Agent
# Netzwerksocket schließen
# Kill: Archivierer
# Kill: Logger
# Message Queue (Logger) löschen
# Mutex-Down: Tafel
# Freigabe: Shared Memory (Tafel)
\end{lstlisting}

\subsubsection{Login-Thread}
\begin{lstlisting}
# Warte auf Login von Client
# Mutex-Down für Zugriff auf Client-Liste 
    # Prüfung: Clientname bereits in Liste?
        # wenn Ja: Fehlermeldung an Client
    # Schreibrecht: Nein
    # Wenn Rolle Dozent: Prüfung, bereits ein Dozent angemeldet?
        # wenn Ja: ändere Rolle Dozent -> Student
        # wenn Nein: Schreibrecht zuweisen
    # Wenn Rolle Tutor: ändere Rolle zu Student
    # (IP/DNS-Name,) Client-Name, Rolle + Zugriffsrecht in Liste eintragen
    # Semaphore (Clientzähler) Up
# Mutex-Up für Zugriff auf Client-Liste
# Starte Client-Thread für neuen Client
# Trigger Broadcasting-Agent: Update Anzahl Clients
\end{lstlisting}

\subsubsection{Client-Threads}
\begin{lstlisting}
# Rückmeldung an Client: Login erfolgreich
# Warte auf Befehle von verbundenem Client
> quit (Client beenden) bzw. Client schließt Verbindung
    # Mutex-Down für Zugriff auf Client-Liste
        # Client aus Client-Liste austragen
        # Semaphore (Client-Zähler) Down
    # Mutex-Up
    # Trigger Broadcasting-Agent: Sende neue Clientanzahl
    # Tread beenden
\end{lstlisting}
\begin{lstlisting}
> request (Schreibrechte anfordern)
    # Mutex-Down für Zugriff auf Client-Liste
        # hat Client bereits schreibrecht bzw. ist Dozent?
            # wenn Ja: Fehlermeldung
        # suche Dozent in Clientliste
            # kein Dozent: Fehlermeldung
    # Mutex-Up
    # Anfrage für Schreibrecht an Dozent
    # Warte auf Antwort von Dozent
        # Antwort Nein: Fehlermeldung an anfragenden Benutzer
    # Mutex-Down für Zugriff auf Client-Liste
        # setze alter Benutzer mit Schreibrechten: keine Schreibrechte
        # aktueller Benutzer: Schreibrecht
    # Mutex-Up
    # Statusänderung alter Benutzer: keine Schreibrechte
    # Statusänderung anfragender Benutzer: Schreibrechte    
    # Trigger Broadcasting-Agent: Tutor = 1
\end{lstlisting}
\begin{lstlisting}
> shutdown
    # Mutex-Down für Zugriff auf Client-Liste
        # Benuter ist Dozent?
            # wenn Nein: Fehlermeldung
    # Mutex-Up
    # Signal senden: System beenden
\end{lstlisting}
\begin{lstlisting}
> release
    # Mutex-Down für Zugriff auf Client-Liste
        # Benutzer ist Tutor?
            # wenn Nein: Fehlermeldung
        # aktueller Benutzer: Schreibrecht > Nein
        # ändere Rolle Tutor -> Student
        # Dozent: Schreibrecht Ja
    # Mutex-Up
    # Trigger Broadcasting-Agent: Tutoren = 0
    # Sende Statusänderung Dozent: Schreibrecht erhalten
\end{lstlisting}
\begin{lstlisting}
> acquire
    # Mutex-Down für Zugriff auf Client-Liste
        # aktueller Benutzer ist Dozent?
            # wenn Nein: Fehlermeldung
        # entziehe Tutor Schreibrecht
        # setze Dozent Schreibrecht
    # Mutex-Up
    # Statusänderung vorheriger Tutor: keine Schreibrechte
    # Trigger Broadcasting-Agent: Tutoren = 0
    # Sende Statusänderung Dozent: Schreibrechte
\end{lstlisting}
\begin{lstlisting}
> modify
    # Mutex-Down für Zugriff auf Client-Liste
        # Benutzer hat Schreibrechte?
            # wenn Nein: Fehlermeldung
    # Mutex-Up
    # Mutex-Down für Zugriff auf Tafel
        # Änderungen in Shared Memory schreiben
    # Mutex-Up
    # Trigger Broadcasting-Agent: Tafeländerung
\end{lstlisting}
\begin{lstlisting}
> clear
    # Mutex-Down für Zugriff auf Client-Liste
        # Benutzer hat Schreibrechte?
            # wenn Nein: Fehlermeldung
    # Mutex-Up
    # Mutex-Down für Zugriff auf Tafel
    # Trigger Archivierer
        *** Im Archivierer ist kein Mutex-Down notwendig. Dies erfolgt vor
        *** der Triggerung des Archivierers, um sicherzustellen, das in der
        *** Zwischenzeit niemand anders auf die Tafel zugreift.
        *** Der Archivierer macht nach dem Sichern der Tafel einen Mutex-Up
    # Mutex-Down für Zugriff auf Tafel
        # Lösche Tafel  
    # Mutex Up
    # Trigger Broadcasting-Agent: Tafel leer
\end{lstlisting}

\subsubsection{Broadcasting-Agent}
\begin{lstlisting}
# Warte auf Trigger
# Wenn Tafeländerung
    # Mutex-Down für Zugriff auf Tafel
        # Lese Tafelinhalt
    # Mutex-Up
# Sende Nachricht an alle verbundenen Clients
\end{lstlisting}


\subsection{Funktionalität Logger}
\begin{lstlisting}
# Öffne Message Queue
# Warte auf Messages
# Schreibe Zeitstempel + Nachricht in Datei (zeilenweise)
\end{lstlisting}


\subsection{Funktionalität Archivierer}
\begin{lstlisting}
# Öffne Logfile (schreibbar)
# Warte auf Trigger bzw. Ablauf von Timer (Debug-Modus)
# Ausgelöst durch Timer?
    # wenn Ja: Mutex
    *** Vor dem Auslesen der Tafel ist nur ein Mutex-Down notwendig, wenn
    *** der Auslöser für die Archivierung durch den Timer erfolgt ist.
    *** Andernfalls ist dies bereits durch den Client-Thread geschehen um
    *** sicherzustellen, dass der Tafelinhalt erst nach dem Archivieren
    *** gelöscht wird.
# Tafel auslesen
# Mutex-Up
# Zeitstempel + Tafelinhalt in Datei schreiben (blockweise)
\end{lstlisting}


\subsection{Synchronisationsprotokoll}
siehe Petrinetz.

\subsection{Netzwerkschnittstellen}

\subsubsection{Header}
\myfigure{Header}{header.pdf}
\begin{tabular}{ll}
Type & uint8\_t \\
Lenght & uint16\_t
\end{tabular}

\subsubsection{Login}
\myfigure{Login}{login.pdf}

\subsubsection{Tafelinhalt}
\myfigure{Tafelinhalt}{board_content.pdf}

\subsubsection{Tafel löschen}
\myfigure{Tafel löschen}{clear_board.pdf}

\subsubsection{Schreibrechtanfrage}
\myfigure{Schreibrechtanfrage C -> S}{write_request.pdf}

\myfigure{Schreibrechtanfrage S -> D}{write_requested.pdf}

\myfigure{Schreibrechtanfrage D -> S}{request_reply.pdf}

\subsubsection{Beenden}
\myfigure{Beenden}{shutdown.pdf}

\subsubsection{Status-Nachricht}
\myfigure{Status-Nachricht}{status_nachricht.pdf}

\subsubsection{Fehlernachricht}
\myfigure{Fehlernachticht}{error_message.pdf}